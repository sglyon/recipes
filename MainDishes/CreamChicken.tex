%!TEX root = ../SkousenLyonCookbook.tex

\begin{recipe}
    [ % You can use as many of these as you want. If you don't want to use them just leave it blank like we did going from calory to source.
    % On the bakingtermperature line just change the number.
    % the portion line is for hoy  many people it is supposed to feed
    % the source line is where the recipe came from.
        preparationtime = {15 min},
        bakingtime={20 min},
        bakingtemperature,
        portion,
        calory,
        source = Kelli Skousen
    ]
    % Put the name of the recipe in between the { } on the next line
    {Cream Chicken}
    \index{chicken}
    \index{quick}
    \index{basic}

        \graph
        {% pictures I will help you figure out how to put them in here. Just ask me.
            small,
            big
        }

        \ingredients
        {% List ingredients here. You will have to follow this format exactly: number & unit % item. For example, a cup of butter would be written "1 cup & butter"
        % Also keep in mind that you need to have the \ at the end of all lines except the last one.
            1 can & chicken (see tip)\\
            2 cans & cream of chicken soup\\
            2 c. & Cooked rice
        }

        \preparation{
        % This is where you list the steps. Just start each step with a new line (press enter on keyboard) and the phrase \step. That's it.
            \step If using chicken breasts, cook them first.
            \step Start steaming your choice of brown or white rice.
            \step Cook the chicken and soup together in a sauce pan over medium-high heat.
            \step Serve the soup/chicken mix over the rice.
        }

        % Here you can add an optional tip someone might want to keep in mind while making this. Not sure why we have to call it hint, but we do.
      \hint{You can also use 2-3 cubed chicken breasts. Also, the amount of rice depends on how many people will be eating it - use your best judgment!}
\end{recipe}
\newpage
