%!TEX root = ../SkousenLyonCookbook.tex

% Cranberry Chicken
\begin{recipe}
    [ % You can use as many of these as you want. If you don't want to use them just leave it blank like we did going from calory to source.
    % On the bakingtermperature line just change the number.
    % the portion line is for hoy  many people it is supposed to feed
    % the source line is where the recipe came from.
        preparationtime,
        bakingtime,
        bakingtemperature,
        portion = {\portion{5-6}},
        calory,
        source = Kelli Skousen
    ]
    % Put the name of the recipe in between the { } on the next line
    {Cranberry Chicken}
    \index{chicken}
    \index{basic}

        \graph
        {% pictures I will help you figure out how to put them in here. Just ask me.
            small,
            big
        }

        \ingredients
        {% List ingredients here. You will have to follow this format exactly: number & unit % item. For example, a cup of butter would be written "1 cup & butter"
        % Also keep in mind that you need to have the \ at the end of all lines except the last one.
            5-6 & Chicken Breasts\\
            1  & Russian Salad Dressing\\
            1 package & Lipton Onion soup mix\\
            1 can & Jellied Cranberry Sauce
        }


        \preparation{
        % This is where you list the steps. Just start each step with a new line (press enter on keyboard) and the phrase \step. That's it.
            \step Mix salad dressing, onion soup, and cranberry sauce.
            \step Place some sauce in crockpot to cover the bottem.  Place chicken in crockpot and cover with remaining sauce
            \step Cook on low for 3-4 hours
        }

        % Here you can add an optional tip someone might want to keep in mind while making this. Not sure why we have to call it hint, but we do.
      \hint{Serve with angel hair pasta which you have buttered and covered with parmasan cheese.  Depending on your crockpot and how much chicken, cooking time will vary.  Check chicken after 2 hours to make sure the chicken is not dry.  I usually use frozen chicken.}
\end{recipe}
\newpage
