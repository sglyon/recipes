%!TEX root = ../SkousenLyonCookbook.tex

\begin{recipe}
    [ % You can use as many of these as you want. If you don't want to use them just leave it blank like we did going from calory to source.
    % On the bakingtermperature line just change the number.
    % the portion line is for hoy  many people it is supposed to feed
    % the source line is where the recipe came from.
        preparationtime,
        bakingtime={2-4 h},
        bakingtemperature,
        portion,
        calory,
        source = Cindy Himes
    ]
    % Put the name of the recipe in between the { } on the next line
    {Chicken and Noodles}
    \index{chicken}
    \index{basic}
        \graph
        {% pictures I will help you figure out how to put them in here. Just ask me.
            small,
            big
        }

        \ingredients
        {% List ingredients here. You will have to follow this format exactly: number & unit % item. For example, a cup of butter would be written "1 cup & butter"
        % Also keep in mind that you need to have the \ at the end of all lines except the last one.
            4-6 & Chicken breasts\\
            1 & Italian Seasoning packet\\
            1 can & Cream of chicken soup\\
            4 oz. & Cream cheese\\
            1 pack & Egg noodles\\
            & Salt \& pepper
        }

        \preparation{
        % This is where you list the steps. Just start each step with a new line (press enter on keyboard) and the phrase \step. That's it.
            \step Place the chicken in crock pot and sprinkle with Italian seasoning.
            \step Pour the cream of chicken soup over the chicken.
            \step Cook on low for 2-4 hours (more chicken requires more time).
            \step 30 minutes before servings mix the cream cheese into the crock pot.
            \step Shred the chicken.
            \step When ready to serve cook the egg noodles as directed on the package.
        }

        % Here you can add an optional tip someone might want to keep in mind while making this. Not sure why we have to call it hint, but we do.
      \hint{This meal is great with Cholula.}
\end{recipe}
\newpage
