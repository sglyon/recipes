%!TEX root = ../SkousenLyonCookbook.tex

% Chicken Fajitas
\begin{recipe}
    [ % You can use as many of these as you want. If you don't want to use them just leave it blank like we did going from calory to source.
    % On the bakingtermperature line just change the number.
    % the portion line is for hoy  many people it is supposed to feed
    % the source line is where the recipe came from.
        preparationtime,
        bakingtime={4 h},
        bakingtemperature,
        portion = {\portion{6-8}},
        calory,
        source = Jill Lyon
    ]
    % Put the name of the recipe in between the { } on the next line
    {Chicken Fajitas}
    \index{chicken}
    \index{mexican}
    \index{basic}

        \graph
        {% pictures I will help you figure out how to put them in here. Just ask me.
            small,
            big
        }

        \ingredients
        {% List ingredients here. You will have to follow this format exactly: number & unit % item. For example, a cup of butter would be written "1 cup & butter"
        % Also keep in mind that you need to have the \ at the end of all lines except the last one.
            3-4 & Chicken Breasts\\
            1 &  Onion\\
            2 packages & Fajita mix\\
            1 can & Green Chili Enchilada sauce \\
            1  & green pepper\\
            1  & red pepper
        }


        \preparation{
        % This is where you list the steps. Just start each step with a new line (press enter on keyboard) and the phrase \step. That's it.
            \step Put chicken breasts, onions, enchilada sauce, fajita mix in Crockpot.  Cook on low for 2-3 hours.
            \step Two hours before desired finished time, shred chicken and put in sliced peppers
            \step Serve with warmed tortillas
        }

        % Here you can add an optional tip someone might want to keep in mind while making this. Not sure why we have to call it hint, but we do.
        \hint{Add more of everything if serving more people.  When cooking for two, cut recipe in half except for I usually use most of an onion and most of the bell peppers.  Use half a can of enchilada sauce (we like Hatch Green Chile Enchilada sauce  with roasted garlic).}
\end{recipe}
\newpage
