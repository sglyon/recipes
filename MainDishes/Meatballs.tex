%!TEX root = ../SkousenLyonCookbook.tex

\begin{recipe}
    [ % You can use as many of these as you want. If you don't want to use them just leave it blank like we did going from calory to source.
    % On the bakingtermperature line just change the number.
    % the portion line is for hoy  many people it is supposed to feed
    % the source line is where the recipe came from.
        preparationtime = {25 min},
        bakingtime={30-40 min},
        bakingtemperature={350\degf},
        portion,
        calory,
        source = Shannen Lyon
    ]
    % Put the name of the recipe in between the { } on the next line
    {Meatballs}
    \index{sausage}

        \graph
        {% pictures I will help you figure out how to put them in here. Just ask me.
            small,
            big
        }

        \ingredients
        {% List ingredients here. You will have to follow this format exactly: number & unit % item. For example, a cup of butter would be written "1 cup & butter"
        % Also keep in mind that you need to have the \ at the end of all lines except the last one.
            1 lb. & sausage\\
            1/2 c.       & quick oats\\
            2 Tbsp. & onions, grated\\
            1/2 tsp. & salt\\
            2/3 c. & milk
        }


        \preparation{
        % This is where you list the steps. Just start each step with a new line (press enter on keyboard) and the phrase \step. That's it.
            \step In a bowl, combine ingredients and roll into meatballs.
            \step Brown in large saucepan.
            \step Place in a casserole dish and pour spaghetti sauce over meatballs.
            \step Bake at 350 degrees for 30-40 minutes.
        }

        % Here you can add an optional tip someone might want to keep in mind while making this. Not sure why we have to call it hint, but we do.
      \hint{You can also use hamburger or ground turkey instead of sausage for meatballs.}
\end{recipe}
\newpage
