%!TEX root = ../SkousenLyonCookbook.tex

\begin{recipe}
    [ % You can use as many of these as you want. If you don't want to use them just leave it blank like we did going from calory to source.
    % On the bakingtermperature line just change the number.
    % the portion line is for hoy  many people it is supposed to feed
    % the source line is where the recipe came from.
        preparationtime,
        bakingtime={10-12 min},
        bakingtemperature={375 \degf},
        portion,
        calory,
        source = Jill Lyon
    ]
    % Put the name of the recipe in between the { } on the next line
    {Hersey Kiss Cookies}
    \index{chocolate}
    \index{cookie}

        \graph
        {% pictures I will help you figure out how to put them in here. Just ask me.
            small,
            big
        }

        \ingredients
        {% List ingredients here. You will have to follow this format exactly: number & unit % item. For example, a cup of butter would be written "1 cup & butter"
        % Also keep in mind that you need to have the \ at the end of all lines except the last one.
            1/2 c. & butter\\
            1/2 c. & shortening\\
            1 c. & sugar\\
            1 c. & brown sugar\\
            2 & eggs\\
            1/2 c. & Creamy peanut butter\\
            4 Tbsp. & Milk\\
            2 tsp. & baking soda\\
            1 tsp. & salt\\
            2 tsp. & vanilla\\
            3 1/2 cups & flour\\
            2 large & bags Hershey kisses
            }


        \preparation{
        % This is where you list the steps. Just start each step with a new line (press enter on keyboard) and the phrase \step. That's it.
            \step Roll the dough into little balls an then roll in sugar.
            \step Place the balls on cookie sheets and bake at 375 for 10*12 minutes.
            \step Remove from the oven and place an unwrapped Hershey kiss in the center of the cookie.
        }

        % Here you can add an optional tip someone might want to keep in mind while making this. Not sure why we have to call it hint, but we do.
      \hint{This makes a lot, so treat it as if it were doubled.}
\end{recipe}
\newpage
