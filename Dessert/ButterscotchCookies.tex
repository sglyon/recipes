%!TEX root = ../SkousenLyonCookbook.tex

\begin{recipe}
    [ % You can use as many of these as you want. If you don't want to use them just leave it blank like we did going from calory to source.
    % On the bakingtermperature line just change the number.
    % the portion line is for hoy  many people it is supposed to feed
    % the source line is where the recipe came from.
        preparationtime,
        bakingtime={8-10 min},
        bakingtemperature={375 \degf},
        portion,
        calory,
        source = Jill Lyon
    ]
    % Put the name of the recipe in between the { } on the next line
    {Butterscotch Cookies}
        \index{butterscotch}
        \index{cookie}
        \graph
        {% pictures I will help you figure out how to put them in here. Just ask me.
            small,
            big
        }

        \ingredients
        {% List ingredients here. You will have to follow this format exactly: number & unit % item. For example, a cup of butter would be written "1 cup & butter"
        % Also keep in mind that you need to have the \ at the end of all lines except the last one.
            1/2 c. & butter\\
            1/2 cup & shortening\\
            3/4 c.& brown sugar\\
            1/2 c. & sugar\\
            1 & egg\\
            1 tsp. & baking soda\\
            2 tsp. & vanilla\\
            1/2 tsp. & salt\\
            1 1/2 c. & flour\\
            2 c. & corn flakes\\
            1 c. & oatmeal\\
            1 c. & butterscotch chips
        }


        \preparation{
        % This is where you list the steps. Just start each step with a new line (press enter on keyboard) and the phrase \step. That's it.
            \step Mix all together.
            \step Bake at 375 for 8-10 minutes.
        }

        % Here you can add an optional tip someone might want to keep in mind while making this. Not sure why we have to call it hint, but we do.
      \hint{These are Spencer's favorite cookies!}
\end{recipe}
\newpage
